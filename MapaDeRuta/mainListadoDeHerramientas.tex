\documentclass{article}
\usepackage[utf8]{inputenc}
\usepackage[spanish]{babel}
\usepackage{listings}
\usepackage{graphicx}
\graphicspath{ {images/} }
\usepackage{cite}

\begin{document}

\begin{titlepage}
    \begin{center}
        \vspace*{1cm}
            
        \Huge
        \textbf{Listado de herramientas}
            
        \vspace{1 cm}
        \LARGE
        Informática 2
        \vspace{5cm}
            
        \textbf{Maria Fernanda Tasco Alquichire}
            
        \vfill
            
        \vspace{0.8cm}
            
        \Large
        Despartamento de Ingeniería Electrónica y Telecomunicaciones\\
        Universidad de Antioquia\\
        Medellín\\
        Diciembre de 2021
            
    \end{center}
\end{titlepage}

\tableofcontents
\newpage
\section{Sección introductoria}\label{intro}
En este documento se presenta el porcentaje de dominio que tiene el estudiante de las herramientas principales que se usarán a lo largo del curso de informatica 2. El porcentaje se dividirá en fortalezas y debilidades, así mismo se hace una breve descripción de cada una de ellas.

\section{Herramientas} \label{contenido}
{
Debido a que el curso se trabajarán temas como conceptos básicos de c++, funciones, arreglos, punteros, POO e interfaz gráfica, se decide trabajar en Qt Creator que es ID multiplataforma para programar en c++ plano y también usar la interfaz gráfica. Por otro lado se usará las herramientas Git y GitHub para llevar un control de los programa que se realicen y compartirlos de una manera más efecientes a los docentes. Para finalizar también se desea registrar en texto y en video algunos de los proyectos, parciales y trabajos de investigación de la materia, por lo cuál será importante el buen manejo de Latex, como herramiente editora de texto, y la gestión de videos, de buena calidad, tanto en la preparación, mientras que se hace el video, la ideción y la subida a alguna plataforma.
}
\subsection{Qt Creator}
Fortalezas (20 porciento): 
{
\begin{itemize}
\item{Buen manejo y orden en los archivos, su creación, ubicación y seguimiento, lo cuál también ayuda a la hora del manejo de la herramienta de Git y GitHub.}
\end{itemize}
}

Debilidades (80 porciento):
{
\begin{itemize}
\item{No tengo un buen manejo del Debuger por parte de la experiencia debido a que en las 3 instalaciones que he tenido del programa no ha funcionado}
\item{En ocasiones que he buscado librerias propias de Qt, se me ha dificultado entenderlas y seguir una linea, porque una me lleva a la otra y así y me perdió del objetivo principal por el cuál iba a buscar una librería de Qt}
\item{Siguiendo con las librerias se me ha dificultado encontrar sitios con buenos ejemplos de las librerias, y la hora de intentar implementarlas no me dan el resultado que esperaba}
\item{También me gustaría aprender a llevar un mejor manejo de git en Qt, el profesor en algunas clases a mostrado en la barra de la derecha un seguimiento de varios proyectos y accede a ellos fácilmente}

\end{itemize}
}

\subsection{Git}
Fortalezas (50 porciento):
{
\begin{itemize}
\item{Buen manejo y orden de los archivos y carpetas, en su ubicación y contenido para hacerles seguimiento}
\item{Manejo promedio (50 porciento) del uso de Git por medio de la consola, lo cual me ha permitido entender mejor lo que estoy haciendo, a la hora de crear un control para un proyecto }
\end{itemize}
}
Debilidades (50 porciento):
{
\begin{itemize}
\item{No sé como crear ramas para el mejor control de las versiones de los proyectos. Así mismo no entiendo cómo subir los cambios de las ramas a la rama principal}
\item{Debido a que es un sistema de control de versiones, aún no entiendo como revisar las versiones anteriores}

\end{itemize}
}

\subsection{GitHub}
Fortalezas (30 porciento): 
{
\begin{itemize}
\item{Orden en las carpetas y archivos de los repositorios.}
\item{Buen manejo de los tokens, para configurarlos y generarlos }
\end{itemize}
}
Debilidades (70 porciento): 
{
\begin{itemize}
\item{No he entendido por completo el uso del gitNore y creo que a veces aunque lo incluya en mis proyectos no me ignora los archivos que debe ignorar y no entiendo porque}
\item{Debido a que trabajo por consola, a veces se me ha dificultado el proceso del Fork copiar el código de alguien (solo una vez intente trabajar en equipo) pero fue un fracaso; porque a la hora de hacer el Pull compartir los cambios con la otra persona, los cambios no llegaban, y nunca se pudocer Merge hacer la fución de esos cambios}
\item{}
\end{itemize}
}
\subsection{Latex}
Fortalezas (60 porciento):
{
\begin{itemize}
\item{Las plantillas son intuitivas de entender y aplicar}
\item{Hay variedad de ejemplos y hasta el momento he podido entenderlos y aplicarlos}
\end{itemize}
}
Debilidades (40 porciento):
{
\begin{itemize}
\item{En algunas ocasiones no he usado bien los comandos y trato de modificarlos o colocarlos a modo que se vea bien en el pdf, pero me sale error en el Source, y no sé porque.}
\end{itemize}
}
\subsection{Gestión de videos}
Fortalezas (40 porciento):
{
\begin{itemize}
\item{Orden en los archivos}
\item{Creatividad}
\end{itemize}
}
Debilidades (60 porciento):
{
\begin{itemize}
\item{Aún no he encontrado un buen editor en linux}
\item{No tengo mucha agilidad}
\item{Soy un poco timida, y eso me lleva a procrastinar a la hora de hacer videos con la ayuda de otras personas}
\end{itemize}
}
 
\end{document}

