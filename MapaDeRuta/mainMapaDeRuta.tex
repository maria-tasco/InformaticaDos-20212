\documentclass{article}
\usepackage[utf8]{inputenc}
\usepackage[spanish]{babel}
\usepackage{listings}
\usepackage{graphicx}
\graphicspath{ {images/} }
\usepackage{cite}

\begin{document}

\begin{titlepage}
    \begin{center}
        \vspace*{1cm}
            
        \Huge
        \textbf{MAPA DE RUTA PARA RESOLVER PROBLEMAS}
            
        \vspace{0.5cm}
        \LARGE
        
            
        \vspace{5cm}
            
        \textbf{Maria Fernanda Tasco Alquichire}
            
        \vfill
            
        \vspace{0.8cm}
            
        \Large
        Despartamento de Ingeniería Electrónica y Telecomunicaciones\\
        Universidad de Antioquia\\
        Medellín\\
        Diciembre de 2021
            
    \end{center}
\end{titlepage}

\tableofcontents
\newpage
\section{Sección introductoria}\label{intro}
En este documento se presenta una alternativa de solución de problemas en general, la cuál tiene un objetivo de guiar al usuario a tener una ruta o un mapa para saber por donde empezar, consideraciones a tener, condiciones, preguntas que hacerse.

\section{Sección de contenido} \label{contenido}
La solución propuesta tiene 4 pasos \cite{EDEGE}. a seguir que nos llevaran a profundizar en el análisis de la solución de un problema, a demás de algunas herramientas.
\subsection{Comprensión del problema}
Leer, identificar los datos, reconocer las incognitas, determinar la relación de los datos, Dividir el problema en partes escenciales. Para este paso y el siguiente es importante tener una hoja y lapiz a la mano, para hacer una lluvia de ideas, hacer diagramas de flujo; estas son algunas herramientas para no dejar las ideas al aire y poder darles una estructura clara y eficaz.
\subsection{Planificación de los pasos a seguir}
Con el problema divido en partes, nos preguntamos si el problema es similar a uno que ya hayamos resuelto antes. Para plantear la solución de un problema es importante revisar los conocimientos previos que podrían aportar. Con esto en mente revisar qué datos vamos a usar en una solución (Recordando que dividimos el problema en distinntas partes) y tener en cuenta cómoo cambian para un uso posterior
\subsection{Ejecución del plan}
Comprobar cada uno de los pasos que se van a ejecurtar, para esto primero nos preguntamos si es el caso ¿Qué consigo con cada una de las operaciones/Decisiones? segundo acompaña cada operación o decisión de lo qué haces y con que fin. También es importante tener en cuenta las condiciones externas, esto nos ayudarán a planificar, para prevenir errores o dificultades a futuro. Si hay un  estancamiento, vuelve al primer paso para re-análizar el problema, reordenar las ideas, qué tan eficiente es y si es el caso plantear una nueva estrategía, este último nos podría llevar a investigar herramientas que nos conocemos, o a preguntar a personas que saben del tema.
Por último pero no menos importante, observar el resultado de cada operación/ Decisión.

\subsection{Supervisión}
Leer de nuevo el enunciado y revisar si se ha cumplido con lo que se esperaba, en todos los aspectos, condiciones y contexto. 
¿La solución a la que hemos llegado es lógica, posible y eficiente?

\bibliographystyle{IEEEtran}
\bibliography{references}

\end{document}

