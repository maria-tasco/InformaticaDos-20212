\documentclass{article}
\usepackage[utf8]{inputenc}
\usepackage[spanish]{babel}
\usepackage{listings}
\usepackage{graphicx}
\graphicspath{ {/home/mafeta/Imágenes/parcial1}}
\usepackage{cite}

\begin{document}

\begin{titlepage}
    \begin{center}
        \vspace*{1cm}
            
        \Huge
        \textbf{Implementación Parcial 1}
            
        \vspace{0.5cm}
        \LARGE
        Sistema de desencriptación 
            
        \vspace{5cm}
            
        \textbf{Maria Fernanda Tasco ALquichire}
            
        \vfill
            
        \vspace{0.8cm}
            
        \Large
        Despartamento de Ingeniería Electrónica y Telecomunicaciones\\
        Universidad de Antioquia\\
        Medellín\\
        Febrero de 2022
            
    \end{center}
\end{titlepage}

\tableofcontents
\newpage

\section{Objetivos}\label{intro1}
\subsection{generales}
resumen - abstrac
\subsection{Especificos}
introducion

\section{Sección introductoria}\label{intro2}
\subsection{Resumen}
resumen - abstrac
\subsection{Introdución}
introducion

\section{Aplicación en tinkercad}\label{practica}
En el proyecto de  \textbf{tinkercad } \cite{implementacion} se tuvo en cuenta estos aspectos:





(\ref{intro1}),(\label{intro2}), (\label{investigación}), (\label{practica}), (\label{conclusiones}) y (\label{practica})
\bibliographystyle{IEEEtran}
\bibliography{references}

\end{document}